// src/components/QrScanner.js

import React, { useState } from 'react';
// นำเข้า db และฟังก์ชันที่จำเป็นจากไฟล์ตั้งค่า Firebase ของคุณ
import { db, ref, push, set, serverTimestamp } from '../Firebase/Firebase'; 
import QrReader from 'react-qr-reader';

function QrScanner() {
  const [isScanning, setIsScanning] = useState(false); // ควบคุมการเปิด/ปิดกล้อง
  const [scanResult, setScanResult] = useState(null);
  const [isSaving, setIsSaving] = useState(false); // ควบคุมสถานะการบันทึก

  const handleError = (err) => {
    console.error("เกิดข้อผิดพลาดจากกล้อง:", err);
  };

  const saveToDatabase = async (qrCodeData) => {
    setIsSaving(true);
    
    try {
        // 1. กำหนด Path ที่จะเก็บข้อมูลใน Realtime DB (เช่น โหนด 'scans')
        const dbRef = ref(db, 'scans'); 
        
        // 2. สร้าง Reference ใหม่ด้วย push() เพื่อให้ได้คีย์ ID ที่ไม่ซ้ำกันโดยอัตโนมัติ
        const newRecordRef = (dbRef); 
        
        // 3. บันทึกข้อมูล
        await set(newRecordRef, {
            data: qrCodeData,
            scanned_at: serverTimestamp(), 
            // สามารถเพิ่มข้อมูลอื่น ๆ เช่น user_id, location, etc.
        });
        
        console.log('บันทึกข้อมูล QR Code สำเร็จ:', qrCodeData);
    } catch (error) {
        console.error('เกิดข้อผิดพลาดในการบันทึก:', error);
    } finally {
        setIsSaving(false);
    }
  };

  const handleScan = (data) => {
    if (data) {
      setIsScanning(false); // สแกนสำเร็จ ปิดกล้อง
      setScanResult(data);
      saveToDatabase(data); // บันทึกข้อมูลที่ได้
    }
  };

  return (
    <div style={{ padding: '20px', textAlign: 'center' }}>
      <h1>เครื่องสแกน QR Code</h1>
      
      {/* ปุ่มควบคุมการเริ่มสแกน */}
      <button 
        onClick={() => setIsScanning(true)} 
        disabled={isScanning || isSaving}
        style={{ padding: '10px 20px', fontSize: '16px', cursor: 'pointer' }}
      >
        {isScanning ? 'กำลังรอการสแกน...' : 
         isSaving ? 'กำลังบันทึก...' : 'เริ่มสแกน QR Code'}
      </button>

      {/* Component QrReader จะถูก Render เมื่อ isScanning เป็น true เท่านั้น */}
      {isScanning && (
        <div style={{ width: '300px', margin: '20px auto', border: '1px solid #ccc' }}>
          <QrReader
            delay={300} // หน่วงเวลา 300ms ระหว่างการสแกน
            onError={handleError}
            onScan={handleScan}
            style={{ width: '100%' }}
            // เลือกว่าจะใช้กล้องหน้า (user) หรือกล้องหลัง (environment)
            constraints={{ facingMode: 'environment' }} 
          />
        </div>
      )}
      
      {/* แสดงผลการสแกน */}
      <p style={{ marginTop: '20px' }}>
        ข้อมูลที่สแกนได้: <strong>{scanResult || 'รอการเริ่มสแกน...'}</strong>
      </p>
    </div>
  );
}

export default QrScanner;