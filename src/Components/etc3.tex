// src/components/QrScanner.js

import React, { useState, useEffect } from 'react';
import QrReader from 'react-qr-reader'; 
import { db, ref, set, update, serverTimestamp, onValue } from '../Firebase/Firebase'; 

function QrScanner({ lockerId, actionType, onActionComplete }) {
  const [isScanning, setIsScanning] = useState(false); 
  const [scanResult, setScanResult] = useState(null);
  const [isSaving, setIsSaving] = useState(false);
  const [lockerData, setLockerData] = useState({});
  const [stopScanner, setStopScanner] = useState(false); 

  // --- Listener เฉพาะ Locker ที่เลือก ---
  useEffect(() => {
      if (!lockerId) return;
      const dbRef = ref(db, `lockers/${lockerId}`);
      const unsubscribe = onValue(dbRef, (snapshot) => {
          setLockerData(snapshot.val() || {});
      });
      return () => unsubscribe();
  }, [lockerId]);

  // --- Effect เพื่อรีเซ็ตสถานะการหยุดหลังจากสแกนสำเร็จ ---
  useEffect(() => {
      if (stopScanner) {
          const timer = setTimeout(() => {
              setStopScanner(false);
          }, 500); 
          return () => clearTimeout(timer);
      }
  }, [stopScanner]);

  const handleError = (err) => {
    console.error("เกิดข้อผิดพลาดจากกล้อง:", err);
  };

  const handleValidationAndSave = async (ownerId) => {
      setIsSaving(true);

      try {
          // ⭐ FIX: กำหนดสถานะปัจจุบัน โดยถือว่าไม่มีข้อมูล = 'AVAILABLE' เสมอ
          const currentStatus = lockerData.status || 'AVAILABLE';

          if (actionType === 'DEPOSIT') {
              // 🔴 กรณีเก็บรองเท้า (Deposit)
              if (currentStatus === 'AVAILABLE') {
                  const updates = {
                      status: 'OCCUPIED',
                      ownerId: ownerId, 
                      deposit_time: serverTimestamp(),
                      // ⭐ NEW: เพิ่มสถานะ Relay 1 = สั่งเปิดประตู
                      relay_command: 1 
                  };
                  await update(ref(db, `lockers/${lockerId}`), updates);
                  alert(`✅ เก็บสำเร็จ! สั่งเปิด Locker ${lockerId} เพื่อฝากรองเท้า`);
                  onActionComplete(lockerId, true); 
              } else {
                 // ถ้าสถานะเป็น OCCUPIED (จากการตรวจสอบจริง)
                 alert(`❌ ล็อกเกอร์ ${lockerId} ไม่ว่าง! (ถูกจองแล้ว)`);
                 onActionComplete(lockerId, false, true); // กลับไปหน้าเลือก Locker
              }
          } else if (actionType === 'WITHDRAW') {
              // 🟢 กรณีนำออก (Withdrawal)
              if (currentStatus === 'OCCUPIED' && lockerData.ownerId === ownerId) {
                  const updates = {
                      status: 'AVAILABLE',
                      ownerId: null, // ล้าง ownerId
                      withdrawal_time: serverTimestamp(),
                      // ⭐ NEW: เพิ่มสถานะ Relay 1 = สั่งเปิดประตู
                      relay_command: 1
                  };
                  await update(ref(db, `lockers/${lockerId}`), updates);
                  alert(`✅ ยืนยันสำเร็จ! สั่งเปิด Locker ${lockerId} เพื่อนำรองเท้าออก`);
                  onActionComplete(lockerId, true); 
              } else {
                  alert('❌ ยืนยันไม่สำเร็จ! รหัสไม่ตรงกับเจ้าของล็อกเกอร์');
                  onActionComplete(lockerId, false); 
              }
          }
      } catch (error) {
          console.error("Firebase Update Error:", error);
          alert("เกิดข้อผิดพลาดในการบันทึกข้อมูล (อาจเป็นปัญหาด้านสิทธิ์)");
          onActionComplete(lockerId, false);
      } finally {
          setIsSaving(false);
      }
  };

  const handleScan = (data) => {
    if (data) {
      setScanResult(data);
      handleValidationAndSave(data); 

      setStopScanner(true); 
      setIsScanning(false); 
    }
  };

  // --- FIX: ฟังก์ชันเริ่มสแกนด้วย setTimeout (แก้ปัญหา Play Interruption) ---
  const startScanning = () => {
    setScanResult(null); 
    setTimeout(() => {
        setIsScanning(true);
    }, 1); 
  };

  return (
    <div style={{ padding: '20px', textAlign: 'center' }}>
      <h2>2. สแกน QR Code (LINE ID)</h2>
      <h3>กำลังดำเนินการ: **{actionType}** สำหรับ Locker **{lockerId}**</h3>

      <button onClick={startScanning} disabled={isScanning || isSaving || stopScanner}
        style={{ padding: '10px 20px', fontSize: '16px', cursor: 'pointer', margin: '20px 0' }}
      >
        {isScanning ? 'กำลังรอการสแกน...' : isSaving ? 'กำลังตรวจสอบ...' : 'เริ่มสแกน QR Code'}
      </button>

      {isScanning && !stopScanner && (
        <div style={{ width: '300px', margin: '20px auto', border: '2px solid #61dafb' }}>
          <QrReader key="scanner-active" delay={100} onError={handleError} onScan={handleScan}
            style={{ width: '100%' }} constraints={{ facingMode: 'environment' }} 
          />
        </div>
      )}
      
      <p style={{ marginTop: '20px' }}>สถานะ: **{isSaving ? 'กำลังตรวจสอบ...' : scanResult ? 'สแกนสำเร็จ' : 'รอการสแกน'}**</p>
      
      {scanResult && <p>ค่าที่สแกนได้: {scanResult}</p>}
      
      <button onClick={() => onActionComplete(lockerId, false, true)} style={{ marginTop: '20px' }}>
          กลับไปหน้าเลือกล็อกเกอร์
      </button>
    </div>
  );
}

export default QrScanner;